\documentclass{article}
\usepackage{makeidx}  % allows for index generation
\usepackage{graphicx}
\usepackage{mathptmx}
%  \usepackage{latexsym}
\usepackage{color}   
\usepackage{hyperref}
\hypersetup{ pdftitle={Planets Technical Architecture}, pdfauthor={A.N.Jackson et
al.}, colorlinks=true, linkcolor=black, filecolor=black, pagecolor=black,
urlcolor=black, citecolor=black }
\begin{document}

\title{Planets Technical Architecture}

\author{Andrew N. Jackson}


\maketitle

\section{Planets Technical Architecture}
Planets (Preservation and Long-term Access through NETworked Services) is a four
year project which began in June 2006, co-funded by the Planets Consortium and
the European Union under the Sixth Framework Programme. The project aims to
provide a modular software suite capable of addressing the core digital
preservation challenges that libraries, archives and the digital preservation
community are facing \cite{af-planets-2007}. There are two main end-user
applications, the Planets Preservation Planning Tool (PLATO) and the Planets
Testbed.  These two components address two different but complimentary
preservation contexts.

PLATO \cite{plato2008} provides a guided decision-support environment for
building preservation plans.  These plans are intended to cater explicitly to the
needs of a particular institution, and focus on comparing preservation strategies
(e.g. comparing migration paths).  The consequences of those strategies are
judged by the specific subjective needs of the individual or institution
involved, based on an analysis of how those strategies perform upon a sample of
digital objects from the institutions collection.

In contrast, the Testbed does not attempt to directly capture institutional
needs. Rather, it provides a uniform environment in which the preservation
services the Planets project has developed can be explored and tested. The
Testbed standardises the way these different services are experienced, and
facilitates the measurement of various properties of these services and of the
digital objects they act upon. The Testbed provides a consistent semantic
structure in which these measured properties can be recorded, and this structure
is designed to avoid capturing any subjective judgements concerning the
`significance' of these properties. This divorces the results from the concerns
of any particular preservation context, and so makes it easier for users to share
results across different experiments and between institutions. This structure
also permits meaningful aggregation of results over very many experiments, so
that robustness and performance statistics concerning particular services might
be objectively determined.

This aggregated data can then be disseminated more widely via the Planets tool
registry, currently being developed as an extension of PRONOM \cite{pronom}. 
This is intended to act as a central point of reference for performance data
concerning preservation tools and services. In this way, the Testbed results can
inform the preservation planning process, for example by acting as input to the
PLATO planning workflow.

\subsection{The Interoperablility Framework}
The Testbed is built upon the Planets Interoperability Framework (IF), which is
being presented in detail in a separate paper at this conference.   The IF
provides the infrastructure, concepts and components needed to perform digital
preservation tasks.  It is based on a instance of the JBoss 4.2 enterprise
application platform \cite{sw-jboss}, customised to provide support for the
development of digital preservation applications in a web-based environment.  The
framework provides the core infrastructure require to build a service-oriented
architecture based on the Java EE 5 \cite{sw-jee5} and JAX-WS \cite{sw-jaxws}
technology stacks. Building on this infrastructure, the IF has developed a range
of common components to be shared between preservation applications. These are
standardised across the Planets projects, forming the basis of the interoperable
web service architecture and also providing useful functionality to assist in the
development of application software.

Of crucial importance to this service-oriented approach is the standardisation of
service interfaces for basic preservation actions, called `Level One Services'.
These are attempts to concretely define the basic nouns and verbs of digital
preservation, in the hope of providing a suite of core preservation operations
that can be combined to perform a wide range of preservation tasks. These
definitions are provided in the form of Java interfaces, and any service
developer needs only to implement a single interface in order to define a
Planets-compatible preservation service.

The service interfaces are design to ensure that the services are consistent,
stateless and atomic. This means that Planets services are easy to swap or
combine, and makes it simpler to create software that is capable of invoking many
different preservation tools. For example, chains of migrations can be built up
from individual migration services, and any migration service that conforms to
the Migration interface can be invoked using exactly the same client code. In the
same way, the task of building suitable user interfaces for performing these
tasks becomes a lot more manageble, as a new user interface is only required for
each service interface, and not for each tool.

The most important `noun' at this level is the Digital Object - a concrete Java
class that wraps up the concept of a single digital entity. A Digital Object may
be composed of one or more bytestreams, and these bytestreams can be referred to
by URL, embedded as byte arrays, or as streamed over SOAP connections via the
MTOM binary data transmission standard \cite{std-mtom}. The Digital Object
provides space for basic metadata, like repository URL and format, and also for
arbitrary tagged metadata chunks. This form avoids any need to explicitly
prescribe the nature of the high-level metadata, while still allowing such
metadata to be associated with an object. Roughly speaking, a Digital Object can
be mapped to the PREMIS notion of a Representation object \cite{std-premis}, and
does not encapsulate intellectual entities like `Journal', `Article' or
`Collection', as the precise definition of such terms can vary between contexts
and institutions. As far as the level-one services are concerned, these are
abstract notions, and the basic preservation operations are concerned only with
simple concrete digital objects, such as `A PDF File'.

The notion of file format, or rather of Digital Object format, is dealt with by
adopting a simple scheme for specifying the format in terms of a URI.  When used
for preservation purposes, these URIs are expressed in the form of PRONOM IDs
(e.g. \emph{info:pronom/fmt/18}, meaning PDF 1.4). However, this fine-grained
system does not supply a simple way of stating concepts like `all PDF versions',
and so an additional URI scheme based on file extensions is also supported (e.g.
\emph{planets:fmt/ext/pdf}). The technical registry provides the means of
translation between these schemes, based on the information in the DROID
signature file \cite{sw-droid,ws-pronom}.

\subsection{Preservation Services}
The most important `noun' for the level-one services is the Digital Object, which
is a concrete Java class that wraps up the concept of a single digital entity.
Roughly speaking, a Digital Object can be mapped to the PREMIS notion of a
Representation object \cite{std-premis}. Digital Object Format is dealt with by
adopting a simple URI specification, based on the PRONOM format identifier
scheme\footnote{\url{http://www.nationalarchives.gov.uk/pronom/}}. Using this
noun and a few other concepts, the IF describes some core preservation `verbs' in
terms of standardised Java interfaces.  Planets services are written as
implementations of these interfaces, allowing different tools to be used
transparently when performing the same conceptual operation.

\subsubsection{Identify}
An operation that takes a Digital Object as input, and returns a list of matching
format URIs and the methodology used to determine the format (e.g. signature
matching, partial parse, full parse etc).
DROID\footnote{\url{http://droid.sourceforge.net/}} and
JHOVE\footnote{\url{http://hul.harvard.edu/jhove/}} are examples of tools that
have been wrapped to provide this functionality.

\subsubsection{Validate} 
An operation that takes a Digital Object and a format URI as input, and which
returns a report that states whether the object is well-formed and valid with
respect to the indicated format. Examples include JHOVE.

\subsubsection{Characterise}
\label{xcl}
This more general interface allows two operations: the first takes a format URI
as the input and returns a list of measurable properties. The second expects a
Digital Object as input and returns a list of extracted properties and values.

Examples include the New Zealand Metadata
Extractor\footnote{\url{http://meta-extractor.sourceforge.net/}} and JHOVE. The
issue of characterisation also is addressed by another Planets sub-project - the
eXtensible Characterisation Languages
(XCL\footnote{\url{http://planetarium.hki.uni-koeln.de/public/XCL/}}), XCDL and
XCEL \cite{xcl2008,becker-xcdl}. XCDL provides an abstract representation of the
content of digital objects, whereas XCEL describes the way in which the
properties of a particular digital object format can be extracted and mapped into
an XCDL description. This functionality has been made available through the
Planets Testbed as a `Characterise' service.

\subsubsection{Compare Properties}
This operation takes two sets of Digital Object properties and returns a detailed
comparison of the degree of equivalence between those properties.

\subsubsection{Migrate}
\label{ss-pa}
A preservation action specified by a Digital Object, a source format URI and a
target format URI. When called, the service attempts to transform the Digital
Object from the source to the target format, and then returns the result as a new
Digital Object. Planets has created migration services for a wide range of
digital object formats including images, text and document formats, and video and
audio files.  The Testbed also provides access to the SIARD database migration
service, which is capable of migrating databases in proprietary formats to a more
accessible XML format \cite{heusher-siard}.


\subsection{Characterisation Services}
Given these nouns, we can now describe some of the preservation verbs. The
Planets Preservation Characterisation (PC) team has worked to define standard
interfaces for identification, validation, characterisation and comparison of
digital objects. For example, the concept of identification is expressed as a
Java interface with a single method that takes a Digital Object as input, and
returns a list of matching format URIs as output. The interface specification
provides space for the identification service to report the methodology used to
determine the format (e.g. signature matching, partial parse, full parse etc), as
well as a more generic service report (where general information, warnings and
errors can be logged). Given this definition, multiple different identification
tools, like DROID and JHOVE, can all be wrapped behind a common fa\c{c}ade.  A
similar Validation interface has also been defined, based a method that takes a
Digital Object and a format URI as input, and which returns a report that states
whether the object is well-formed and valid with respect to the indicated format.

The third and more general Characterisation interface allows two operations: the
first takes a format URI as the input and returns a list of measurable
properties. The second expects a digital object and possible optional service
specific configuration parameters as input and returns a list of extracted
properties.  The fourth characterisation interface is Comparison, where two sets
of digital object properties can be compared in order to judge how similar they
are.


\subsubsection{The eXtensible Characterisation Languages}
\label{xcl}
In prior work, detailed characterisation of digital objects has been carried out
largely by hand, for example in the Dutch Testbed \cite{potter-2002}. This is a
error prone and time-consuming activity and therefore infeasible for mass
experiment scenarios as the Planets Testbed that involves processing and
evaluating huge collections of data. This issue is addressed by the
\emph{eXtensible Characterisation Languages (XCL)} \cite{xcl2008} and its tools -
one of the key concepts and technologies that is being developed within Planets.
The XCL specification \cite{std-xcl} consists of two XML meta-languages, XCDL and
XCEL. XCDL provides an abstract representation of the content of digital objects,
wheras XCEL describes the way in which the properties of a particular digital
object format can be extracted and mapped into an XCDL description. Currently
there are XCEL descriptions available for the formats PNG, TIFF, GIF, WAV and PDF
and partly DOCX).

To support automated validation and quality assessment of object conversions the
XCL delivers a set of tools \cite{becker-xcdl} which have been wrapped within the
project as two different level-one Planets services. The first implements
Characterise interface described above, and uses the XCL Extractor and File
Property Metrics (FPM) tools.  The latter provides the means to list all of the
properties that are relevant to a particular format, and the former is used to
actually perform the extraction of those properties.

The second service is based on the XCL Comparator tool, and forms the inspiration
for the Comparison service interface. This tool defines property specific
definitions of metrics and their implementation as algorithms to judge on the
equality of two XCDL documents, and proscribes a standard format for the results
of such comparisons.

\subsubsection{Testbed Properties}
\label{ss-planprop}
The Testbed employs an ontology for the formal representation of digital object
and service properties. The starting point is provided by the \emph{OWL Lite}
\cite{w3c-owl-ref} \emph{XCLOntology}, used by the XCL team to represent the
properties of digital objects and the relations between them, and where it is
used to facilitate the automated property comparison and experiment evaluation.

The \emph{Testbed ontology} extends this core model and re-uses these property
definitions in a broader context. The Testbed ontology contains a generic class
hierarchy of the concepts that are used for creating structured and measurable
evaluation criteria. Testbed ontology uses the concepts of behavior, appearance,
content, structure, and context \cite{rothbik1999} as a generic classification
scheme for property individuals. The Testbed ontology also introduces digital
object properties that the XCL tools cannot determine, and properties of other
entities, like services.

The process of defining the measurable properties of digital objects, files, and
Planets services is crucial not only for the Testbed, but across the whole
Planets project. Driven by Testbed requirements the Planets project has
established a digital object properties working group (DOPWG) that is focusing on
exploring existing initiatives in the field of significant properties as InSPECT
\cite{inspect} and on integrating these property schemes into the Testbed.

\subsection{Preservation Actions}
\label{ss-pa}
The IF defines the Migration service interface via a single operation with three
required arguments: a Digital Object, a source format URI and a target format
URI. A fourth argument is optional, and allows the caller to specify one or more
service-specific invocation parameters. When called, the service attempts to
transform the Digital Object from the source to the target format, and then
returns the result as a new Digital Object.

Emulation provides a suitable environment for interpreting a digital object, and
this can used to render the object directly (so that the user may interact with
it), or utilised to implement a migration. From the Planets perspective, an
automatic (i.e. requiring no human intervention) migration performed via
emulation is just another migration service. Therefore, such a service would
implement the same interface and be called in the same way as any other
migration.  The fact that emulation is used `under the hood' only appears in the
service metadata.

The second case, where an emulation session is provided so that the user can
access the digital object directly, is currently being explored by the IF, PA and
Testbed teams.  Further information about the state of this work will be given in
section \ref{current-emul}.

\subsection{The Workflow Engine}
\label{ss-wee}
Building on the atomic low-level Planets level-one service interfaces the IF
provides a workflow system consisting of a workflow interface description, a
surrounding framework (registry, factory) and a corresponding execution engine.
The workflow framework defines Planets level-two services in terms of workflow
templates. The templates's structure defines the overall workflow in terms of the
level-one service interface types to operate upon and provides the surrounding
orchestration and decision making process (looping, branching, sequencing,
exception handling, transactions).  The template parameters define the particular
services that will be invoked and the digital objects that should be processed.
The workflow execution engine maintains a batch queue of workflows with
parameters, and attemps to ensure that contention for resources is minimised
while each one is running.

These level-two services do not have to be standardised across Planets, and are
instead intended to conform to a particular context, e.g. to meet the needs of a
certain institution or application. This allows higher-level metadata management
issues to be cleanly separated from the level-one preservation services, which
can be reused across contexts without placing limitations or particular
expectations on the way each content holder works.

\subsection{Perisistance \& Integration}
Planets is not building an archive, and all of the level-one Planets services are
stateless.  Therefore, in order to have Digital Objects to pass to the services,
and in order to be able to store the results, we require some kind of standard
storage interface. The IF provides a standard Data Registry interface for this
purpose, and the Testbed experiment workflows use this interface to read Digital
Objects out of any of the available repositories, and when the experiment has
been executed, write results back again.


\subsection{About properties and comparison}
Note that our Characterise and Compare interfaces are closely tied to the XCL
tools, and is generally responsible for their form.  For example, there is no

Compare(c1,c2,comparisonPropertyList) => characteristics3

interface, because this is currently impossible in XCL.  There are no
'super-classes' for properties, as they are only defined at the format level.
There may be e.g. an 'byteorder' property for PNG, and a 'big endian flag' in
JPG, but only the comparator service itself knows they are comparable.  The XCL
ontology will contain the fact that they are comparable instances of an abstract
'endianness' property, but at the moment there is no 'endianness' property that
can be applied across formats.

Essentially, the XCL effort splits the notion of digital object properties into
three parts:

1. The definition of properties, for particular formats (XCEL). 2. The values of
those properties, for a particular instance (XCDL). 3. The relationships between
those properties (XCL Ontology \& the Comparator)

Progress is more rapid on 1 and 2 because these are well-defined tasks - it is
clear how to proceed once the appropriate standard or reference implementation
has been chosen.  The third item is more difficult to proceed with, as the best
choice of abstract super-class is often subjective.  Should it be 'endianness',
with a enum of options?  Or a more flexible 'byteOrder = badc' specifier that
covers all the weird 'middle-endian' cases?  Furthermore, these higher-level
definitions must be mapped to actual bit sequences, and in general this is not
practical using only an ontology of first-order logic.  You need a
Turing-complete system to translate representations, and this is why the final
logic is in the Comparator tool itself, not the ontology.

I hope that makes some sense.  Perhaps I should take this and try to write it up
in a clearer form?

In short, the 'optimal' ontology tends to depend upon the size of the set of
concrete entities that it attempts to encapsulate.





The Migrate interface does take a list<Parameter> parameter also that serves
exactly the purpose you suggest.  Additionally an action from the Service Dev
Workshop is that all interfaces will now take list<Parameter> as a parameter,
Identify and Validate didn't.

The Characterise interface has been more awkward as it has been too closely tied
to the XCDL tool at times.  Currently this interface provides two methods
Characterise(format) => property list Characterise(object, list<Parameters>) =>
characteristics

The use of list<Parameters> rather than list<Properties> was discussed at the
Workshop and there is still a set of changes to be made regarding use of
properties.  Fabian is looking at this and will be producing a proposal shortly
(within days).  We do envisage the functionality you specify.

I'm guessing that the paper doesn't make this clear.  I hope this puts your
mind at rest.



I was reading the TB ECDL paper and it raised some questions about the service
interfaces.
 
Chacterise appears to be Chacterise(format) => property list Characterise(object)
=> characteristics (prop/val pairs)
 
I would have expected: 

Characterise(object,properties) => characteristics

(prop/val pairs for the specified properties) 

e.g.

Characterise(pdf123,{pageCount,imageCount,allFontsEmbedded})
 
Compare appears to be 

Compare(characteristics1, characteristics2) => characteristics3 

where characteristics3 is a set of prop/val pairs such as
wordVectorDistance=18 lineCountMatch=90\% and so on. I would have expected:
Compare(c1,c2,comparisonPropertyList) => characteristics3 where the
comparisonPropertyList is something like {wordVectorDistance,lineCountMatch}
 
Without providing this additional information, the services are doomed to extract
every possible characteristic.  The Jhove interface has already shown that this
is a terrible idea!
 
A similar comment for Migrate. Migrate(srcObject,srcFormat,tgtFormat) really
needs some sort of information about the migration process.  Consider a migration
from TIFF to J2K - what are the conversion parameters?  What sort of compression?
 Without some additional information, we will need many many services - one for
each needed combination of parameters.  This sounds pretty awkward.

\subsection{Planets Properties [SS/AL]}
\label{ss-planprop}
A key challenge in digital preservation is to preserve a digital object's
appearance, behaviour, quality and usability over time and within changing
technical environments. The data-centric preservation approach addresses this by
keeping the data usable over time and therefore requires a meaningful
understanding of an object's significant properties. Identifying, measuring and
evaluating properties and their extracted values takes a central role within the
Testbed's experimentation methodology.

[

The Testbed employs an ontology for the formal representation of digital object
and Planets service properties. These properties play a central role in the
automatic measurement one the one hand, and the manual evaluation of criteria in
the step 3 of the Testbed experiment process on the other hand.

]

[TODO AL: mention: hard to establish significant properties that's why the TB is
going for the reuse of existing sets]

[ANJ: Move this lot up to earlier in the story?]

[Note: does anybody know initiatives we could link to regarding definition of
significant property sets]

The Testbed employs an ontology for the formal representation of digital object
service properties. \emph{OWL Lite} \cite{w3c-owl-ref} is the ontology language
that is used for representing the properties and the relations between them. The
reason for choosing this form of representation is twofold: Firstly, ontologies
are highly flexible and extensible. The ontology language OWL provides the
language-construct \emph{owl:imports} which is clearly intended for reuse and
extensibility of existing ontologies. Secondly, ontology reasoning technologies
can be used to make implicit relations and assertions explicit. OWL Lite as well
as the more expressive OWL DL are restricted subsets of the OWL ontology
languages family and have an important advantage when it comes to applying
reasoning algorithms: In contrary to OWL Full they are decidable, and there are
reasoners like RACER, Fact++, or Pellet \cite{huangliyang2008} that allow to make
use of the actual potential of OWL.

[ANJ: Move the above test to earler in the paper?

[SS to ANJ: Feel free to do it!]

The question what the properties of digital objects, files, and Planets services
actually are, is crucial not only for the Testbed, but is discussed Planets-wide
and beyond that in the context of long term preservation of electronic content in
general. Driven by Testbed requirements the Planets project has established a
digital object properties working group (DOPWG) that is focusing on exploring
existing initiatives in the field of significant properties as InSPECT
\cite{inspect} [Note from AL: does anybody know additional significant property
initiatives? SS: I found some sources of information here:
http://www.dpconline.org/graphics/events/080407workshop.html] and on integrating
results in a Testbed usable way. Currently the Testbed has been focusing on
integrating the \emph{XCLOntology} (see section \ref{xcl}) to support automated
property comparison and experiment evaluation. While this ontology is mainly
designed to model the knowledge domain of measurable and in principle extractable
properties of specific file-formats and their uniform notation as XCL-properties,
the \emph{Testbed ontology} extends this model on the conceptual as well as on
the instances level. On the conceptual level, the Testbed ontology contains a
generic class hierarchy of the concepts that are used for creating structured and
measurable evaluation criteria (used in step 3-6 of the Testbed experiment
process). And on the instances level, the Testbed ontology introduces
non-extractable properties as individuals which are not contained in the
XCLOntology but are usable for providing different browsable view representations
of a given set of properties. The Testbed ontology is an extension of the
XCLOntology while selected property instances of the XCLOntology are going to be
mapped into the class hierarchy of the Testbed ontology. For example, among other
additional concepts, the Testbed ontology uses the concepts behavior, appearance,
content, structure, and context \cite{rothbik1999} as a generic classification
scheme for property individuals. In a collaborative manner the property
individuals might be strictly categorized or loosely assigned to one or more of
these concepts depending on the logical constraints that have been defined for
the OWL classes that represent these concepts.


\section{Tools}
The initial set of characterisation tools planned for deployment are shown in
table \ref{table-pc-s}, and the preservation action tools are shown in table
\ref{table-pa-s}.  The latter includes another Planets software project, the
SIARD database migration tool, which is capable of migrating propriatery
databases to a more accessible XML format \cite{heusher-siard}. All of these
tools will be accessible via the public Testbed, allowing a wide range of
migration pathways and different preservation strategies to be explored.

\begin{table}
\centering
\begin{tabular}{|l|l|l|}
  \hline
  Tool  & Tool Type & Service Interfaces \\
  \hline
  Droid & Java & Identify\\
  Fine Free File & CLI & Identify\\
  JHOVE 1 & Java & Identify, Characterise, Validate\\
  libtiff & JNI & Identify, Characterise\\
  New Zealand Metadata Extractor & Java & Identify, Characterise\\ 
  ODF Toolkit & Java & Validate\\
  pngcheck & CLI & Characterise \\
  XCL Comparator & CLI & Compare \\
  XCL Extractor & CLI & Characterise \\
  \hline
\end{tabular}
\caption{A summary of the characterisation tools available through the Planets Testbed. The tool types are as follows: 'Java' is pure Java code, 'CLI' is a wrapped command line interface and 'JNI' is a Java Native Interface wrapped as a service.}
\label{table-pc-s}
\end{table}

\begin{table}
\centering
\begin{tabular}{|l|l|l|}
  \hline
  Tool  & Tool Type & Supported Formats \\
  \hline
  abiword & CLI & DOC, HTML, PDF, RTF, TXT \\
  avidemux & CLI & MPEG, AVI \\
  dialogika & WS & DOC, DOCX \\
  dvips & CLI & DVI, PostScript \\
  gimp & CLI & GIF, EPS, JPEG, PNG, PS, TIFF, BMP \\
  imagemagick & JNI & JPEG, JP2, TIFF, GIF, BMP, PCX, TGA, PDF  \\
  jasper19 & CLI & JPEG, JP2 \\
  Java SE 1.5 & Java (built-in) & GIF, BMP, PNG, JPEG  \\
  JJ2000 & Java & PPM, PGM, JP2  \\
  jtidy & Java &  HTML, XHTML \\
  mdb2siard & CLI & MDB, SIARD \\
  msgtext & CLI & Email MSG, Text \\
  netPBM & CLI & JPEG, PNM \\
  openXML & CLI & DOC, DOCX \\
  Open Office & Server  & DOC, ODF, PDF, PDF/A \\
  openjpeg & CLI & TIFF, JP2 \\
  pdf2ps & CLI & PDF, PS \\
  pdfbox & Java  & PDF, HTML, Text \\
  ps2pdf & CLI & PDF, PS \\
  Apache Sanselan & Java  & PNG, GIF, TIFF, BMP, PBM, PPM \\
  sox & CLI  & MP3, AIFF, FLAC, OGG, WAV\\
  \hline
\end{tabular}
\caption{Overview of the migration tools that will be deployed on the Testbed, with an indication of the input and/or output formats the services that wrap those tools provide.  Tool types are as in table \ref{table-pc-s}, with two additional types: `WS' means that this is a third-party external web service that has been wrapped as a Planets-compliant service. `Server' means that this service makes a tool-specific socket connection to a different kind of server in order to peform the migration.}
\label{table-pa-s}
\end{table}

\subsection*{Acknowledgements}
The Planets Testbed Research and Development work presented within this paper is partially supported by European Community under the Information Society Technologies (IST) Programme of the 6th FP for RTD - Project IST-033789.


\bibliographystyle{plain}
\bibliography{pta-refs}

\end{document}

